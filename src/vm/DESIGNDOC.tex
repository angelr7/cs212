            +---------------------------+
            |          CS 212           |
            | PROJECT 3: VIRTUAL MEMORY |
            |      DESIGN DOCUMENT      |
            +---------------------------+

---- GROUP ----

>> Fill in the names and email addresses of your group members.
Angel Ruiz <ar7@stanford.edu>
Pilli Cruz-De Jesus <pilli@stanford.edu>
Fabian Luna <luna1206@stanford.edu>

---- PRELIMINARIES ----

>> If you have any preliminary comments on your submission, notes for the
>> TAs, or extra credit, please give them here.

>> Please cite any offline or online sources you consulted while
>> preparing your submission, other than the Pintos documentation, course
>> text, lecture notes, and course staff.

            PAGE TABLE MANAGEMENT
            =====================

---- DATA STRUCTURES ----

>> A1: Copy here the declaration of each new or changed `struct' or
>> `struct' member, global or static variable, `typedef', or
>> enumeration.  Identify the purpose of each in 25 words or less.

---- ALGORITHMS ----

>> A2: In a few paragraphs, describe your code for accessing the data
>> stored in the SPT about a given page.

>> A3: How does your code coordinate accessed and dirty bits between
>> kernel and user virtual addresses that alias a single frame, or
>> alternatively how do you avoid the issue?

Only access through user 

---- SYNCHRONIZATION ----

>> A4: When two user processes both need a new frame at the same time,
>> how are races avoided?

---- RATIONALE ----

>> A5: Why did you choose the data structure(s) that you did for
>> representing virtual-to-physical mappings?

               PAGING TO AND FROM DISK
               =======================

---- DATA STRUCTURES ----

>> B1: Copy here the declaration of each new or changed `struct' or
>> `struct' member, global or static variable, `typedef', or
>> enumeration.  Identify the purpose of each in 25 words or less.

---- ALGORITHMS ----

>> B2: When a frame is required but none is free, some frame must be
>> evicted.  Describe your code for choosing a frame to evict.

>> B3: When a process P obtains a frame that was previously used by a
>> process Q, how do you adjust the page table (and any other data
>> structures) to reflect the frame Q no longer has?

>> B4: Explain your heuristic for deciding whether a page fault for an
>> invalid virtual address should cause the stack to be extended into
>> the page that faulted.

---- SYNCHRONIZATION ----

>> B5: Explain the basics of your VM synchronization design.  In
>> particular, explain how it prevents deadlock.  (Refer to the
>> textbook for an explanation of the necessary conditions for
>> deadlock.)

>> B6: A page fault in process P can cause another process Q's frame
>> to be evicted.  How do you ensure that Q cannot access or modify
>> the page during the eviction process?  How do you avoid a race
>> between P evicting Q's frame and Q faulting the page back in?

>> B7: Suppose a page fault in process P causes a page to be read from
>> the file system or swap.  How do you ensure that a second process Q
>> cannot interfere by e.g. attempting to evict the frame while it is
>> still being read in?

>> B8: Explain how you handle access to paged-out pages that occur
>> during system calls.  Do you use page faults to bring in pages (as
>> in user programs), or do you have a mechanism for "locking" frames
>> into physical memory, or do you use some other design?  How do you
>> gracefully handle attempted accesses to invalid virtual addresses?

---- RATIONALE ----

>> B9: A single lock for the whole VM system would make
>> synchronization easy, but limit parallelism.  On the other hand,
>> using many locks complicates synchronization and raises the
>> possibility for deadlock but allows for high parallelism.  Explain
>> where your design falls along this continuum and why you chose to
>> design it this way.

             MEMORY MAPPED FILES
             ===================

---- DATA STRUCTURES ----

>> C1: Copy here the declaration of each new or changed `struct' or
>> `struct' member, global or static variable, `typedef', or
>> enumeration.  Identify the purpose of each in 25 words or less.

---- ALGORITHMS ----

>> C2: Describe how memory mapped files integrate into your virtual
>> memory subsystem.  Explain how the page fault and eviction
>> processes differ between swap pages and other pages.

>> C3: Explain how you determine whether a new file mapping overlaps
>> any existing segment.

---- RATIONALE ----

>> C4: Mappings created with "mmap" have similar semantics to those of
>> data demand-paged from executables, except that "mmap" mappings are
>> written back to their original files, not to swap.  This implies
>> that much of their implementation can be shared.  Explain why your
>> implementation either does or does not share much of the code for
>> the two situations.

               SURVEY QUESTIONS
               ================

Answering these questions is optional, but it will help us improve the
course in future quarters.  Feel free to tell us anything you
want--these questions are just to spur your thoughts.  You may also
choose to respond anonymously in the course evaluations at the end of
the quarter.

>> In your opinion, was this assignment, or any one of the three problems
>> in it, too easy or too hard?  Did it take too long or too little time?

>> Did you find that working on a particular part of the assignment gave
>> you greater insight into some aspect of OS design?

>> Is there some particular fact or hint we should give students in
>> future quarters to help them solve the problems?  Conversely, did you
>> find any of our guidance to be misleading?

>> Do you have any suggestions for the TAs to more effectively assist
>> students, either for future quarters or the remaining projects?

>> Any other comments?

            +---------------------------+
            |          CS 212           |
            | PROJECT 3: VIRTUAL MEMORY |
            |      DESIGN DOCUMENT      |
            +---------------------------+

---- GROUP ----

>> Fill in the names and email addresses of your group members.
Angel Ruiz <ar7@stanford.edu>
Pilli Cruz-De Jesus <pilli@stanford.edu>
Fabian Luna <luna1206@stanford.edu>

---- PRELIMINARIES ----

>> If you have any preliminary comments on your submission, notes for the
>> TAs, or extra credit, please give them here.

>> Please cite any offline or online sources you consulted while
>> preparing your submission, other than the Pintos documentation, course
>> text, lecture notes, and course staff.

            PAGE TABLE MANAGEMENT
            =====================

---- DATA STRUCTURES ----

>> A1: Copy here the declaration of each new or changed `struct' or
>> `struct' member, global or static variable, `typedef', or
>> enumeration.  Identify the purpose of each in 25 words or less.

---- ALGORITHMS ----

>> A2: In a few paragraphs, describe your code for accessing the data
>> stored in the SPT about a given page.

>> A3: How does your code coordinate accessed and dirty bits between
>> kernel and user virtual addresses that alias a single frame, or
>> alternatively how do you avoid the issue?

Only access through user ?

---- SYNCHRONIZATION ----

>> A4: When two user processes both need a new frame at the same time,
>> how are races avoided?

?

---- RATIONALE ----

>> A5: Why did you choose the data structure(s) that you did for
>> representing virtual-to-physical mappings?

               PAGING TO AND FROM DISK
               =======================

---- DATA STRUCTURES ----

>> B1: Copy here the declaration of each new or changed `struct' or
>> `struct' member, global or static variable, `typedef', or
>> enumeration.  Identify the purpose of each in 25 words or less.

---- ALGORITHMS ----

>> B2: When a frame is required but none is free, some frame must be
>> evicted.  Describe your code for choosing a frame to evict.

>> B3: When a process P obtains a frame that was previously used by a
>> process Q, how do you adjust the page table (and any other data
>> structures) to reflect the frame Q no longer has?

TODO

>> B4: Explain your heuristic for deciding whether a page fault for an
>> invalid virtual address should cause the stack to be extended into
>> the page that faulted.

TODO

---- SYNCHRONIZATION ----

>> B5: Explain the basics of your VM synchronization design.  In
>> particular, explain how it prevents deadlock.  (Refer to the
>> textbook for an explanation of the necessary conditions for
>> deadlock.)

>> B6: A page fault in process P can cause another process Q's frame
>> to be evicted.  How do you ensure that Q cannot access or modify
>> the page during the eviction process?  How do you avoid a race
>> between P evicting Q's frame and Q faulting the page back in?

>> B7: Suppose a page fault in process P causes a page to be read from
>> the file system or swap.  How do you ensure that a second process Q
>> cannot interfere by e.g. attempting to evict the frame while it is
>> still being read in?

>> B8: Explain how you handle access to paged-out pages that occur
>> during system calls.  Do you use page faults to bring in pages (as
>> in user programs), or do you have a mechanism for "locking" frames
>> into physical memory, or do you use some other design?  How do you
>> gracefully handle attempted accesses to invalid virtual addresses?

---- RATIONALE ----

>> B9: A single lock for the whole VM system would make
>> synchronization easy, but limit parallelism.  On the other hand,
>> using many locks complicates synchronization and raises the
>> possibility for deadlock but allows for high parallelism.  Explain
>> where your design falls along this continuum and why you chose to
>> design it this way.

             MEMORY MAPPED FILES
             ===================

---- DATA STRUCTURES ----

>> C1: Copy here the declaration of each new or changed `struct' or
>> `struct' member, global or static variable, `typedef', or
>> enumeration.  Identify the purpose of each in 25 words or less.

This struct is used to keep track of mapid mappings created from mmap calls.
    struct mapid_elem
    {
        mapid_t mapid;
        int fd;
        void *start_addr;
        struct list_elem elem;
    };

The mapid_list thread member variable keeps track of all of the mapid mappings
created from a process's calls to mmap. The cur_mapid variable is the current
id to give out to the next call of mmap.
    struct thread
    {
       struct list mapid_list; 
       int cur_mapid; 
    }

---- ALGORITHMS ----

>> C2: Describe how memory mapped files integrate into your virtual
>> memory subsystem.  Explain how the page fault and eviction
>> processes differ between swap pages and other pages.

We create entries in our supplemental page table for each page of a memory
mapped file. In the supplemental page table entries for these files,
we keep track of information about the file that we will need to
read from and write to the file on disk, such as the pointer to the file
struct and the offset within the file we want to read and write to, along
with the mapid for the mmap mapping. These spt entries are always writable
and look similar to non memory mapped file entries, except non memory mapped
entries have a mapid of NO_MAPID. This allows us to determine whether an 
entry in the spt belongs to a memory mapped file.

During the page fault and eviction processes, we are able to use the
information stored in the spt entries to know when we are dealing with
a memory mapped file's page. When page faulting, we are able to check the
page's memory flag to see if the page is currently on disk when we are
loading it in. If it is on disk, we know that it comes from a file and we
are able to read page into physical memory from the file on disk. If we
have a memory flag indicating that the page is on the swap partition, we
are able to read that page back in to memory from the swap slot. During 
eviction, we are able to use the mapid page entry member variable to
determine if we are evicting a memory mapped file. If that is the case,
we write that page's contents back to the file on disk. If the page entry's
mapid is equal to the NO_MAPID constant, then we proceed by determing if
we need to evict the page to the swap partition if the page belongs to an
executable but is writable, or is simple a stack page. Else, if the page
belongs to a file and is read-only, we are able to simply zero out the page
since we can bring in the unchanged file from disk when we page fault on
this page again.

>> C3: Explain how you determine whether a new file mapping overlaps
>> any existing segment.

Given the mapping's starting address as a parameter, we first check that it
is valid (is multiple of PGSIZE and non-NULL). If it is valid, we proceed by
checking is the new mapping overlaps any existing segment. We use the file's 
length, obtained by calling file_length on the file we are mapping, and
calculate the total number of pages that the file will need to take up in its 
mapping. We loop through all the pages that the mapping would use beginning
from the given starting address parameter and check if
an entry for that page already exists in the supplemental page table. If we
ever encounter an already mapped page, we know that there is overlap and
fail mmap. If we detect that there is no overlap, we are able to create page
entries for all the pages we just looped over during our overlap check.

---- RATIONALE ----

>> C4: Mappings created with "mmap" have similar semantics to those of
>> data demand-paged from executables, except that "mmap" mappings are
>> written back to their original files, not to swap.  This implies
>> that much of their implementation can be shared.  Explain why your
>> implementation either does or does not share much of the code for
>> the two situations.

Because of these similarities, we are able to treat both mmap mappings
and data demand-paged from the executables the same when page faulting.
When we page fault, we simply check if the data we need to bring in is
on disk, which is true in both of these situation, and simple read in
the data from the file on disk utilizing the same page entry member
variables for both. The differences between the page entries for
these different situations are the mapid value and writable bool. For 
memory mapped files, all page entries will have a valid mapid value and
writable will always be true. For data demand-paged from executables, 
the mapid value will be NO_MAPID and the writable bool depends on what
section of data it pertains too (code vs. data). This is important for
page eviction and the freeing of pages upon process exit. If we are 
evicting or freeing a page with a valid mapid, we write the data from
memory back to the file on disk. If there is no valid mapid, then we 
use the writable flag to know how to handle an eviction. If the page is
writable, we write the page to a swap slot and read it back in from the
swap slot when paged back in. If not writable, no writes to disk need to be
made and we can read the data back in from the file system when it is paged
back in. When simple freeing and not evicting page entries with no mapid,
we never write back to disk. We are not able to share much of the code
when evicting or freeing in these two situations because we must write to
different partitions or not write depending on the situation. However,
as explained before, we are able to share much of the code for these two
situations when reading in from files (excluding data evicted to swap).

               SURVEY QUESTIONS
               ================

Answering these questions is optional, but it will help us improve the
course in future quarters.  Feel free to tell us anything you
want--these questions are just to spur your thoughts.  You may also
choose to respond anonymously in the course evaluations at the end of
the quarter.

>> In your opinion, was this assignment, or any one of the three problems
>> in it, too easy or too hard?  Did it take too long or too little time?

>> Did you find that working on a particular part of the assignment gave
>> you greater insight into some aspect of OS design?

>> Is there some particular fact or hint we should give students in
>> future quarters to help them solve the problems?  Conversely, did you
>> find any of our guidance to be misleading?

>> Do you have any suggestions for the TAs to more effectively assist
>> students, either for future quarters or the remaining projects?

>> Any other comments?
